%% -*- show-trailing-whitespace: t -*-
%% gpl2016.tex - Lecture on the FSF GPL Compliance Lab
%%
%% Copyright (C)  2016 Yoni Rabkin
%%
%% Attribution-ShareAlike 4.0 International (CC BY-SA 4.0)


\renewcommand{\familydefault}{cmr}
\newcommand{\nitem}{\item[$\ast$]}
\newcommand{\sitem}{\item[]}
\newcommand{\pp}{\pause}

\documentclass[
  size=12pt,
  style=simple,
  paper=screen
]{powerdot}

\pddefinetemplate{titleslide}{
  titlefont=\large\bfseries\raggedright,
  texthook=tl,textpos={.05\slidewidth,.83\slideheight},
  textwidth=.9\slidewidth,textfont=\tabcolsep0pt,
  textheight=.66\slideheight
}{%
}
\pddefinetemplate{basic}{
  titlepos={.05\slidewidth,.93\slideheight},
  titlewidth=.9\slidewidth,textheight=.66\slideheight,
  titlefont=\large\bfseries\raggedright,
  lfpos={.03\slidewidth,.04\slideheight},
  tocslidesep=.6ex,textheight=.68\slideheight,
  ifsetup=portrait,
  textpos={.05\slidewidth,.83\slideheight},
  textwidth=.9\slidewidth,
  ifsetup=landscape,
  textpos={.2\slidewidth,.83\slideheight},
  textwidth=.75\slidewidth
}{%
  \psline[linewidth=.8pt](0,.9\slideheight)(\slidewidth,.9\slideheight)%
}

\title{FSF GPL Compliance Lab \& Recent Developments in Free Software Licensing}

\author{Yoni Rabkin (yrk@gnu.org)}

\begin{document}

\maketitle

\begin{wideslide}{Introductions}
  \begin{itemize}
    \setlength{\itemsep}{1em}

    \nitem I'm a programmer, GNU maintainer, and a paralegal
    \pp

    \nitem I've been answering people's licensing questions as a
    volunteer at the FSF's GPL Compliance Lab for decade...
    \pp

    \nitem ...but I'm not a lawyer, nor do I represent the FSF
  \end{itemize}
\end{wideslide}

\begin{wideslide}{Abstract}
  Let's talk about:
  \begin{itemize}
    \setlength{\itemsep}{1em}

    \nitem The GPL Compliance Lab's relation to the FSF.

    \nitem What the GPL Compliance Lab does.

    \nitem Recent developments in free software licensing that The Lab
    is involved with.

    \nitem How to get involved.

  \end{itemize}
  \pp

  I would be very grateful for questions, comments, and suggestions at
  any stage of the talk; don't sit on your questions!
\end{wideslide}

\begin{wideslide}{The FSF}
  The Free Software Foundation (FSF) \ldots
  \begin{itemize}
    \setlength{\itemsep}{1em}

    \nitem \ldots is a nonprofit with a worldwide mission to promote
    computer user freedom\pp\\(with a well-defined meaning to the
    word ``freedom'')

    \pp

    \nitem \ldots publishes licenses, such as the GNU GPL

    \pp

    \nitem \ldots leads the GNU Project \pp (2015: 460 projects led by
    volunteers, 321 copyright assignments/disclaimers completed)

    \pp

    \nitem \ldots is effective: tiny by choice, a strong voice,
    efficiently uses a small budget \pp (FY2015: \$1.2m operating
    budget, average donation \$101)

    \pp

    \nitem \ldots inspires a lot of activism
  \end{itemize}
\end{wideslide}

\begin{wideslide}{The GPL Compliance Lab}
  The GPL Compliance Lab is an internal part of the FSF.

  \begin{itemize}
    \setlength{\itemsep}{1em}

    \nitem an informal activity since 1992

    \nitem formally since 2001\\\pp(I started volunteering in 2006)

    \pp

    \nitem run by FSF staff, with the help of volunteers
  \end{itemize}
\end{wideslide}

\begin{wideslide}{The GPL Compliance Lab}
  The Compliance Lab's activities are:
  \begin{itemize}
    \setlength{\itemsep}{1em}

    \nitem \emph{Education \& Support}

    \nitem \emph{Copyright \& Compliance}

    \nitem \emph{Verification \& Certification}
  \end{itemize}
\end{wideslide}

\begin{wideslide}{Education \& Support}
  The GPL FAQ, articles, news posts, as well as answering
  questions\ldots lots of questions at licensing@fsf.org.

  \pp

  \begin{itemize}
    \setlength{\itemsep}{1em}

    \nitem Is this license a free software license, and is it
    GPL-compatible?

    \pp

    \nitem Is this a GPL violation?

    \pp

    \nitem Can we include footage of GPL-licensed software in our
    movie?

    \pp

    \nitem Can I use the GPL to license hardware?

    \pp

    \nitem Can I require that people using my GPL'd work cite me in
    their academic paper?

    \pp

    \nitem May I print this piece of GNU art on underwear I'm selling?
  \end{itemize}
\end{wideslide}

\begin{wideslide}{Education \& Support}
  \begin{itemize}
    \setlength{\itemsep}{1em}

    \nitem Questions from: private persons, companies of all sizes,
    governments, military, lawyers, researchers, artists, etc.

    \pp

    \nitem Proprietary software companies write in all the time \ldots
    \pp but the service is given gratis and at no cost only to the
    free software community.

    \pp

    \nitem A testament to the relevance of the FSF, and to how much
    people trust the FSF.
  \end{itemize}

\end{wideslide}

\begin{wideslide}{Copyright \& Compliance}
  Enforce the terms of licenses for software for which the FSF holds
  the copyright (and help others do the same).

  \pp

  \begin{itemize}
    \setlength{\itemsep}{1em}

    \nitem The FSF gets a lot of violation reports, and investigates
    all of them (128 reports in 2015)

    \pp

    \nitem Collect the who, what, and how of the case.

    \pp

    \nitem Verify that a violation actually occurred.

    \pp

    \nitem Contact the right people.

    \pp

    \nitem The goal is compliance, and repairing damage to the
    community; not money

  \end{itemize}
\end{wideslide}

\begin{wideslide}{Copyright \& Compliance}
  Bringing parties into compliance is best done quietly, quickly, and
  professionally.

  \pp

  \begin{itemize}
    \setlength{\itemsep}{1em}

    \nitem Most parties are unaware of the violation, and act quickly
    to correct it

    \pp

    \nitem A quiet initial contact is usually enough, and the involved
    parties are often please to be back in compliance\pp \\ (the terms of
    GPLv3 make getting back into compliance easier for everyone.)

    \pp

    \nitem In other cases the FSF has expert legal advice from the
    Software Freedom Law Center.

    \pp

    \nitem 6 licensing cases successfully resolved in 2015

  \end{itemize}
\end{wideslide}

\begin{wideslide}{Verification \& Certification}
  Verification \& Certification on the software side of things.
  \pp

  \begin{itemize}
    \setlength{\itemsep}{1em}

    \nitem Maintain a list of committed, freedom respecting GNU/Linux
    distros.

    \pp

    \nitem Work on the Free Software Directory.

  \end{itemize}
\end{wideslide}

\begin{wideslide}{Verification \& Certification}
  On the hardware side of things, the Respects Your Freedom (RYF)
  hardware certification program.
  \pp

  \begin{itemize}
    \nitem Publish and maintain criteria for freedom-respecting
    hardware:

    \pp

    \begin{itemize}
    \item Always 100\% free software \pp
    \item User installation of modified software, compile-able with free software \pp
    \item Software for building, installation and maintenance and
      administration must be free software, free documentation \pp
    \item No spying, cooperation with GNU project policies \pp
    \item Avoids confusion with other products and endorsements \pp
    \item If the device supports encumbered formats, it must also
      support free formats that serve the same purpose. \pp
    \item Relevant patents must be freely licensed \pp
    \item Failure to meet these criteria terminates the endorsement
    \end{itemize}

  \end{itemize}
\end{wideslide}

\begin{wideslide}{Verification \& Certification}
  The Respects Your Freedom (RYF) hardware certification program has been a success.

  \pp

  \begin{itemize}
    \setlength{\itemsep}{1em}

    \nitem Awards certification to products which meet that criteria.

    \pp

    \nitem Already awarded RYF certification to a total of 19 hardware
    products (6 in 2015) \pp with more on the way, and more in the
    formal review process.

  \end{itemize}
\end{wideslide}

\begin{wideslide}{Recent Developments in Free Software Licensing}
  The Compliance Lab is involved in a number of ongoing issues,
  including, but not limited to:
  \begin{itemize}
    \setlength{\itemsep}{1em}

    \nitem \emph{The Linux Kernel and ZFS}

    \nitem \emph{Google v Oracle}

    \nitem \emph{Save Wifi}
  \end{itemize}
\end{wideslide}


\begin{wideslide}{Recent Developments}
  The Linux Kernel and ZFS

  \pp

  \begin{itemize}

    \nitem The kernel Linux is currently under GPLv2.

    \pp

    \nitem The CDDL is a free software license which is
    GPL-incompatible because of its weak, per-file, copyleft.

    \pp

    \nitem ZFS is a file system, now owned by Oracle and licensed
    under the CDDL.

    \pp

  \end{itemize}
  \begin{itemize}

    \nitem ZFS can be integrated via FUSE, which causes no licensing
    issues \ldots \pp But linking ZFS to the kernel Linux via a LKM is
    fraught.

    \pp

    \nitem Debian has recognized, and side-stepped the issue by
    placing ZFS in their \emph{contrib} repository.

    \pp

    \nitem Ubuntu decided to distribute ZFS as a Linux kernel module,
    which is potentially a violation of both the CDDL and GPLv2.

    \pp

    \nitem Oracle can make this whole issue disappear by releasing ZFS
    under a GPL-compatible license.

  \end{itemize}
\end{wideslide}

\begin{wideslide}{Recent Developments}
  Google v Oracle

  \pp

  \begin{itemize}
    \setlength{\itemsep}{0.5em}

    \sitem \emph{Aug 2010} - Oracle sues Google over Google's
    implementation of Java for Android.

    \pp

    \sitem \emph{May 2012} - Original decision ends in ruling that
    APIs are not subject to copyright.

    \pp

    \sitem \emph{May 2014} - Remanded to Federal court on appeal ends
    in ruling that APIs are subject to copyright.

    \pp

    \sitem \emph{Jun 2015} - Supreme court denies the petition to see
    the case.

    \pp

    \sitem \emph{May 2016} - Return to district court for fair use
    trail with a unanimous jury decision that Google's use was fair
    use. \pp \\ \ldots Oracle announces it would appeal.

    \pp

  \end{itemize}

  Software Freedom Law Center and EFF file friend of the court briefs,
  and the Compliance Lab will continue to monitor and report.
\end{wideslide}

\begin{wideslide}{Save Wifi}
  In 2015 the FCC announced the proposal of new rules requiring
  manufacturers to implement locks on all wireless devices. The FSF
  met with the FCC to voice concerns:

  \pp

  \begin{itemize}
    \nitem No clear separation between locked down firmware and the
    OS.

    \pp

    \nitem Users forced to give up control of their wireless devices
    to ``authorized parties''.

    \pp

    \nitem Creates vendor lock-in.

    \pp

  \end{itemize}

  The FCC's response was that the FCC affirms that this is designed to
  prevent the user changing the software even if those changes would
  comply with the rules. Moreover, that the rules would require the
  manufactures to implement security features to prevent modification.
  \pp

  \ldots but the fight continues in 2016.

\end{wideslide}

\begin{wideslide}{Volunteering for the Compliance Lab}
  You too can stare at copyright licenses for hours on end.
  \pp

  \begin{itemize}
    \setlength{\itemsep}{1em}

    \nitem Yes, there is a test.

    \pp

    \nitem There is a training and mentoring period, so long-term
    commitment is best.

  \end{itemize}
\end{wideslide}

\begin{wideslide}{End Matter}
  \begin{itemize}
    \setlength{\itemsep}{1em}

    \sitem Written using \emph{powerdot}
    (\url{www.ctan.org/pkg/powerdot}).

    \sitem Released under the terms of the Attribution-ShareAlike 4.0
    International (CC BY-SA 4.0) license.

    \sitem Source code available at: \url{http://yrk.nfshost.com/repos/gpl2016.git}

  \end{itemize}
\end{wideslide}


\end{document}
\endinput

%% gpl2016.tex ends here.
